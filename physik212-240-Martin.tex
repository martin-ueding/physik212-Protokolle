% Copyright © 2012 Martin Ueding <dev@martin-ueding.de>
%
\documentclass[11pt,german]{article}
\usepackage[T1]{fontenc}
\usepackage[a4paper, left=3cm, right=2cm, top=2cm, bottom=2cm]{geometry}
\usepackage[activate]{pdfcprot}
\usepackage[ngerman]{babel}
\usepackage[parfill]{parskip}
\usepackage[utf8]{inputenc}
\usepackage{amsmath}
\usepackage{amssymb}
\usepackage{amsthm}
\usepackage{color}
\usepackage{epstopdf}
\usepackage{graphicx}
\usepackage{hyperref}
\usepackage{setspace}
\usepackage{units}


\definecolor{darkblue}{rgb}{0,0,.5}

\hypersetup{
	breaklinks=false,
	colorlinks=true,
	linkcolor=black,
	menucolor=black,
	urlcolor=darkblue
}

\setlength{\columnsep}{2cm}

\newcommand{\arcsinh}{\mathrm{arcsinh}}
\newcommand{\asinh}{\mathrm{arcsinh}}
\newcommand{\messwert}{\textcolor{blue}{\square}}
\newcommand{\emesswert}{\left(\messwert \pm \messwert \right)}
\newcommand{\card}{\mathrm{card}}
\newcommand{\diam}{\mathrm{diam}}
\newcommand{\e}[1]{\cdot 10^{#1}}
\newcommand{\fehlt}{\textcolor{red}{Hier fehlen noch Inhalte.}}
\newcommand{\half}{\frac{1}{2}}
\newcommand{\laplace}{\vnabla^2}
\newcommand{\vnabla}{\vec{\nabla}}
\renewcommand{\d}{\, \mathrm d}

\title{physik212 -- Versuch 240 \\ Gleichströme, Spannungsquellen und Widerstände}
\author{Martin Ueding}

\begin{document}

\maketitle

\tableofcontents

\vfill

Stellen in \textcolor{blue}{blau} werden bei der Versuchsdurchführung eingetragen. Zum Beispiel werden Messwerte mit „$\messwert$“ markiert. Stellen in \textcolor{red}{rot} müssen noch vor Versuchsbeginn vervollständigt werden.

\newpage

%%%%%%%%%%%%%%%%%%%%%%%%%%%%%%%%%%%%%%%%%%%%%%%%%%%%%%%%%%%%%%%%%%%%%%%%%%%%%%%
%                                 Einleitung                                  %
%%%%%%%%%%%%%%%%%%%%%%%%%%%%%%%%%%%%%%%%%%%%%%%%%%%%%%%%%%%%%%%%%%%%%%%%%%%%%%%

\section{Einleitung}

\fehlt

%%%%%%%%%%%%%%%%%%%%%%%%%%%%%%%%%%%%%%%%%%%%%%%%%%%%%%%%%%%%%%%%%%%%%%%%%%%%%%%
%                                   Theorie                                   %
%%%%%%%%%%%%%%%%%%%%%%%%%%%%%%%%%%%%%%%%%%%%%%%%%%%%%%%%%%%%%%%%%%%%%%%%%%%%%%%

\section{Theorie}

\fehlt

%%%%%%%%%%%%%%%%%%%%%%%%%%%%%%%%%%%%%%%%%%%%%%%%%%%%%%%%%%%%%%%%%%%%%%%%%%%%%%%
%                                  Aufgaben                                   %
%%%%%%%%%%%%%%%%%%%%%%%%%%%%%%%%%%%%%%%%%%%%%%%%%%%%%%%%%%%%%%%%%%%%%%%%%%%%%%%

\section{Aufgaben}

\subsection{Aufgabe A: ideale Stromquelle}

Eine ideale Stromquelle hat keinen Innenwiderstand.

\subsection{Aufgabe B: Messvorschrift}

Da über dem Innenwiderstand $R_i$ eine Spannung $R_i I$ abfällt, kann
mit dem Strom und der gemessenen Spannung zurückgerechnet werden, groß
die Leerlaufspannung $U_0$ ist. Um den Widerstand zu bestimmen, müssen
allerdings zwei Messungen mit verschiedenen Ströme durchgeführt werden.

Also zweimal $I$ und $U$ messen, dann gilt folgendes Gleichungssystem:
\begin{align*}
	U_0 &= U_1 + I_1 R_i \\
	U_0 &= U_2 + I_2 R_i
\end{align*}

Dies kann dann nach $U_0$ und $R_i$ aufgelöst werden.

\subsection{Aufgabe C: Wheatstonesche Brücke}

Folgende Beziehung soll hergeleitet werden:
\[ R_x = \frac{R_1}{R_2} R_0 \]

Die Idee bei der Wheatstoneschen Brücke ist, dass sich beide Seiten
genau die Waage halten und in der Mitte kein Strom mehr fließt. Das
Potentiometer wird so eingestellt, dass beim drücken des Tasters keine
Ladung fließt. Dies bedeutet, dass über $R_1$ und $R_x$ gleich viel
Spannung abfällt.

Der Strom, der durch die jeweilige Seite fließt, ist:
\begin{align*}
	I_l &= \frac U{R_1 + R_2} \\
	I_r &= \frac U{R_x + R_0}
\end{align*}

Da kein Strom in der Mitte fließt, gilt:
\[ I_r R_1 = I_l R_x \]

Durch Einsetzen erhalte ich:
\begin{align*}
	\frac U{R_1 + R_2} R_1 &= \frac U{R_x + R_0} R_x \\
	\frac {R_1}{R_2}  &= \frac {R_x}{R_0} \\
	R_x &= \frac {R_1}{R_2} R_0
\end{align*}

\subsection{Aufgabe D: Strommessung}

Idealerweise wird das Messgerät knapp voll ausgelastet, 80\% vielleicht.  Somit
soll durch das Messgerät einen Strom von $\unit[0.8]{mA}$ messen. Der restliche
Strom muss über einen anderen Widerstand laufen.

Nach der Stromteilerregel:
\[ I_M + I_S = \unit[4]A \]

\[ I_M = \frac{R_S}{R_S + R_i} I \]

Daraus folgt für den Shunt:
\[ R_S = \frac{I_M R_i}{I - I_m} \approx \unit[0.2]{m\Omega} \]

\subsection{Aufgabe E: Voltmeter}

Das Voltmeter misst letztlich den Strom, der durch es fließt. Dieser ist im
Idealfall proportional zur außen anliegenden Spannung.

Wenn ich $\unit[1]V$ messen möchte, brauche ich einen Widerstand von
$\unit[100]{k\Omega}$, damit über diesem diese Spannung abfällt. Da dieser
Widerstand und der Innenwiderstand parallel geschaltet sind, geht durch jeden
nur noch der halbe Strom durch, die gemessene Spannung ist also nur noch halb
so groß. Das Messgerät sollte dann $\unit[0.5]V$ anzeigen.

\subsection{Aufgabe F: Spannung mit Ampèremeter}

Man kann mit einem Ampèremeter eine Spannung messen, indem man den Strom durch
einen parallelen Widerstand misst. Dieser sollte allerdings sehr groß sein im
Verhältnis zum Verbraucher, dessen Spannungsabfall man bestimmen möchte. Nach
der Stromteilerregel gilt dann:
\[ I_M = \frac{R_M}{R_M + R_V} I \]

Den Strom $I_M$ kann man bestimmen. Wenn man $R_M$ und $R_V$ kennt, lässt sich
daraus $I$ bestimmen. Aus $I$ kann man dann $U = R_V I$ den Spannungsabfall
über dem Verbraucher (ohne Messgerät) bestimmen.

\subsection{Aufgabe G: Formeln}

Seien $R_{U, i}$ und $R_{I, i}$ die Innenwiderstände der Messgeräte und $R_P$
der Widerstand des Potentiometers.
\[ R_A = R_{I, i} + \left( \frac{1}{R_x} + \frac{1}{R_{U, i}} \right)^{-1} + R_P \]
\[ R_B = \left( \frac{1}{R_{I, i} + R_x} + \frac{1}{R_{U, i}} \right)^{-1} + R_P \]

%%%%%%%%%%%%%%%%%%%%%%%%%%%%%%%%%%%%%%%%%%%%%%%%%%%%%%%%%%%%%%%%%%%%%%%%%%%%%%%
%                                Durchführung                                 %
%%%%%%%%%%%%%%%%%%%%%%%%%%%%%%%%%%%%%%%%%%%%%%%%%%%%%%%%%%%%%%%%%%%%%%%%%%%%%%%

\section{Aufbau und Durchführung}

\subsection{Aufgabe a: $U$-$I$-Abhängigkeit}

Wir vermessen die erste Schaltung, Schaltung A.

Als Vorwiderstand für das Voltmeter benutzen wir $\unit[(\messwert \pm
\messwert)] \Omega$ und für das Ampèremeter $\unit[(\messwert \pm \messwert)]
\Omega$.

\begin{center}
	\begin{tabular}{cc}
		$U$ in $\unit V$ & $I$ in $\unit A$ \\
		\hline
		$\messwert$ & $\messwert$ \\
		$\messwert$ & $\messwert$ \\
		$\messwert$ & $\messwert$ \\
			 \vdots & \vdots
	\end{tabular}
\end{center}

Dabei sind die Fehler in der Spannung und Strom:
\[ \Delta V = \unit[\messwert]V, \quad \Delta A = \unit[\messwert] A \]

\subsection{Aufgabe c: $R_x$ mit Multimeter}

Wir bestimmen den Widerstand mit einem Multimeter auf
\[ R_x = \unit[(\messwert \pm \messwert)]{\Omega} \]

\subsection{Aufgabe d: Spannungsteiler}

\label{durchführung-d}

Wir benutzen einen Spannungsteiler an einer $\unit[4]V$ Stromquelle. Dort sind
$R_1$ und $R_2$ beliebig vorgegeben. An den Abnehmer des Spannungsteilers
hängen wir parallel Voltmeter und die Last aus Ampèremeter und Lastwiderstand
$R_a$.

Wir messen mit einem Mavometer für 10 verschiedene Lastwiderstände $R_a$ den
Strom $I$ und die Spannung $U$.

\begin{center}
	\begin{tabular}{ccc}
		$R_x$ in $\unit \Omega$ & $U$ in $\unit V$ & $I$ in $\unit A$ \\
		\hline
		$\messwert$ & $\messwert$ & $\messwert$ \\
		$\messwert$ & $\messwert$ & $\messwert$ \\
		$\messwert$ & $\messwert$ & $\messwert$ \\
		$\messwert$ & $\messwert$ & $\messwert$ \\
		$\messwert$ & $\messwert$ & $\messwert$ \\
		$\messwert$ & $\messwert$ & $\messwert$ \\
		$\messwert$ & $\messwert$ & $\messwert$ \\
		$\messwert$ & $\messwert$ & $\messwert$ \\
		$\messwert$ & $\messwert$ & $\messwert$ \\
		$\messwert$ & $\messwert$ & $\messwert$
	\end{tabular}
\end{center}

Dabei sind die Fehler in der Spannung und Strom:
\[ \Delta V = \unit[\messwert]V, \quad \Delta A = \unit[\messwert] A \]

\textcolor{red}{Mir ist an dieser Stelle nicht klar, wie ich den
Innenwiderstand des Mavometers benutzen soll.}

\subsection{Aufgabe f: Helipot}

\label{durchführung-f}

Wir ersetzen nun den Spannungsteiler mit seinen beiden Widerständen durch ein Helipot.

Nun messen wir für einige Einstellungen $x$ die Spannung. Dafür klemmen wir an den Abgriff des Helipots ein Mavometer als Voltmeter. Das Helipot hat $l = \messwert$ Skalenteilen.

\begin{center}
	\begin{tabular}{cc}
		$x$ & $U_1$ in $\unit A$ \\
		\hline
		$\messwert$ & $\messwert$ \\
		$\messwert$ & $\messwert$ \\
		$\messwert$ & $\messwert$ \\
			 \vdots & \vdots
	\end{tabular}
\end{center}

Wir wiederholen die Messung mit einem Lastwiderstand von $R_a = \unit[20] \Omega$.

\begin{center}
	\begin{tabular}{cc}
		$x$ & $U_1$ in $\unit A$ \\
		\hline
		$\messwert$ & $\messwert$ \\
		$\messwert$ & $\messwert$ \\
		$\messwert$ & $\messwert$ \\
			 \vdots & \vdots
	\end{tabular}
\end{center}

Wir wiederholen die Messung mit einem Lastwiderstand von $R_a = \unit[50] \Omega$.

\begin{center}
	\begin{tabular}{cc}
		$x$ & $U_1$ in $\unit A$ \\
		\hline
		$\messwert$ & $\messwert$ \\
		$\messwert$ & $\messwert$ \\
		$\messwert$ & $\messwert$ \\
			 \vdots & \vdots
	\end{tabular}
\end{center}

Dabei ist der Fehler in der Spannungsmessung für in dieser Aufgabe:
\[ \Delta U_1 = \unit[\messwert] \Omega \]

%%%%%%%%%%%%%%%%%%%%%%%%%%%%%%%%%%%%%%%%%%%%%%%%%%%%%%%%%%%%%%%%%%%%%%%%%%%%%%%
%                                 Auswertung                                  %
%%%%%%%%%%%%%%%%%%%%%%%%%%%%%%%%%%%%%%%%%%%%%%%%%%%%%%%%%%%%%%%%%%%%%%%%%%%%%%%

\section{Auswertung}

\subsection{Aufgabe a: $U$-$I$-Abhängigkeit}

\fehlt

\subsection{Aufgabe b: wahrer Wert $R_x$}

\fehlt

\subsection{Aufgabe e: neue Spannungsquelle}

Mit den Daten aus \ref{durchführung-d} errechne ich hier den Innenwiderstand
$R_i^s$ und Leerlaufspannung $U_0^s$ der neuen Spannungsquelle, die das
Potentiometer einschließt.

Dazu erzeuge ich ein $U$-$I$-Diagramm der Messwerte und fitte diese mit einem linearen Modell:
\[ U = U_0^s - R_i^s I \]

\begin{figure}[h!]
	\centering
	\fehlt
	\caption{$U$-$I$-Diagramm}
\end{figure}

Die Steigung gibt mir $R_i^s$, der Schnittpunkt mit der $U$-Achse gibt mir $U_0^s$. Der Fit liefert:
\[
	U_0^s = \unit[\emesswert] V,
	\quad
	R_i^s = \unit[\emesswert] \Omega
\]

Das Verhältnis von $R_2$ zum Gesamtwiderstand $R_1 + R_2$ soll ja konstant
bleiben. Somit kann man nur beide Widerstände um einen gemeinsamen Faktor
$\lambda \in (0, 1)$ verkleinern:
\[
	R_{1,2} \mapsto \lambda R_{1,2}
	\quad \Rightarrow \quad
	\frac{\lambda R_2}{\lambda R_1 + \lambda R_2} = \frac{R_1}{R_1 + R_2},
	\quad
	\frac{\lambda^2 R_1 R_2}{\lambda R_1 + \lambda R_2} = \lambda R_i^s
\]

Wählt man $\lambda$ allerdings zu klein, kann man mit dem Potentiometer nicht
mehr fein einstellen, außerdem würde die neue Spannungsquelle in sich
kurzgeschlossen.

\subsection{Aufgabe f: Helipot}

Hier werte ich die Daten, die in \ref{durchführung-f} gemessen worden sind, aus. Ich stelle die drei Messreihen in einem $U_1$-$x$-Diagramm dar.

\begin{figure}[h!]
	\centering
	\fehlt
	\caption{$U_1$-$x$-Diagramm}
\end{figure}

\fehlt

\subsection{Aufgabe g: Leistung}

Die Leistung im Widerstand ist:
\[ P(x) = U I = \frac{U^2}R \]

Dies trage ich in einem Plot auf.

\begin{figure}[h!]
	\centering
	\fehlt
	\caption{$P$-$x$-Diagramm}
\end{figure}

Das Maximum liegt bei $x = \messwert$. Für die Widerstände $R_1$ und $R_2$ gilt:
\begin{align*}
	\frac{R_1}{R_1 + R_2} &= \frac{x}{l} \\
	R_1 \left( 1 - \frac{x}{l} \right) &= \frac{x}{l} R_2 \\
		   \left( \frac{l}{x} - 1 \right) &= \frac{R_2}{R_1}
\end{align*}

Für das Maximale $x = \messwert$ gilt dann:
\[ \frac{R_1}{R_2} = \messwert \]

%%%%%%%%%%%%%%%%%%%%%%%%%%%%%%%%%%%%%%%%%%%%%%%%%%%%%%%%%%%%%%%%%%%%%%%%%%%%%%%
%                                  Resultat                                   %
%%%%%%%%%%%%%%%%%%%%%%%%%%%%%%%%%%%%%%%%%%%%%%%%%%%%%%%%%%%%%%%%%%%%%%%%%%%%%%%

\section{Resultat}

\fehlt

%%%%%%%%%%%%%%%%%%%%%%%%%%%%%%%%%%%%%%%%%%%%%%%%%%%%%%%%%%%%%%%%%%%%%%%%%%%%%%%
%                                 Diskussion                                  %
%%%%%%%%%%%%%%%%%%%%%%%%%%%%%%%%%%%%%%%%%%%%%%%%%%%%%%%%%%%%%%%%%%%%%%%%%%%%%%%

\section{Diskussion}

\fehlt

\end{document}

% vim: spell spelllang=de
