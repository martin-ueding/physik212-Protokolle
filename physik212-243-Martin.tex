% Copyright © 2012 Martin Ueding <dev@martin-ueding.de>
%
\documentclass[11pt, ngerman]{article}

\usepackage[a4paper, left=3cm, right=2cm, top=2cm, bottom=2cm]{geometry}
\usepackage[activate]{pdfcprot}
\usepackage[iso]{isodate}
\usepackage[parfill]{parskip}
\usepackage[T1]{fontenc}
\usepackage[utf8]{inputenc}
\usepackage{amsmath}
\usepackage{amssymb}
\usepackage{amsthm}
\usepackage{babel}
\usepackage{color}
\usepackage{commath}
\usepackage{epstopdf}
\usepackage{graphicx}
\usepackage{hyperref}
\usepackage{setspace}
\usepackage{units}

\usepackage[charter]{mathdesign}

\definecolor{darkblue}{rgb}{0,0,.5}
\definecolor{darkgreen}{rgb}{0,.5,0}

\hypersetup{
	breaklinks=false,
	citecolor=darkgreen,
	colorlinks=true,
	linkcolor=black,
	menucolor=black,
	urlcolor=darkblue,
}

\setlength{\columnsep}{2cm}

\DeclareMathOperator{\arcsinh}{arsinh}
\DeclareMathOperator{\arsinh}{arsinh}
\DeclareMathOperator{\asinh}{arsinh}
\DeclareMathOperator{\card}{card}
\DeclareMathOperator{\diam}{diam}

\newcommand{\emesswert}{\left(\messwert \pm \messwert \right)}
\newcommand{\e}[1]{\cdot 10^{#1}}
\newcommand{\fehlt}{\textcolor{red}{Hier fehlen noch Inhalte.}}
\newcommand{\half}{\frac{1}{2}}
\newcommand{\laplace}{\vnabla^2}
\newcommand{\messwert}{\textcolor{blue}{\square}}
\newcommand{\vnabla}{\vec{\nabla}}

\renewcommand{\d}{\, \mathrm d}

\title{physik212 -- Versuch 243 \\ Galvanometer zur Strom- und Ladungsmessung}
\author{Martin Ueding \\ {\small \href{mailto:mu@uni-bonn.de}{mu@uni-bonn.de}}}

\begin{document}

\maketitle

\tableofcontents

\vfill

\begin{small}
	Stellen in \textcolor{blue}{blau} werden bei der Versuchsdurchführung
	eingetragen. Zum Beispiel werden Messwerte mit „$\messwert$“ markiert.
	Stellen in \textcolor{red}{rot} müssen noch vor Versuchsbeginn
	vervollständigt werden.
\end{small}

\newpage

%%%%%%%%%%%%%%%%%%%%%%%%%%%%%%%%%%%%%%%%%%%%%%%%%%%%%%%%%%%%%%%%%%%%%%%%%%%%%%%
%                                 Einleitung                                  %
%%%%%%%%%%%%%%%%%%%%%%%%%%%%%%%%%%%%%%%%%%%%%%%%%%%%%%%%%%%%%%%%%%%%%%%%%%%%%%%

\section{Einleitung}

In diesem Versuch benutzen wir ein Galvanometer. Zuerst betrachten wir nur die
Schwingungen, die es um die Ruhelage vollzieht. Dann messen wir mit einem
Spannungsteiler den Innenwiderstand und die Empfindlichkeit. Zuletzt beschweren
wir das Galvanometer mit Zusatzgewichten und vermessen einen großen Widerstand.

%%%%%%%%%%%%%%%%%%%%%%%%%%%%%%%%%%%%%%%%%%%%%%%%%%%%%%%%%%%%%%%%%%%%%%%%%%%%%%%
%                                   Theorie                                   %
%%%%%%%%%%%%%%%%%%%%%%%%%%%%%%%%%%%%%%%%%%%%%%%%%%%%%%%%%%%%%%%%%%%%%%%%%%%%%%%

\section{Theorie}

Das Drehspulgalvanometer besteht aus einer Spule mit $n$ Windungen, die im
Rechteck $a \times b$ um einen Rotor gewickelt ist. Der Rotor ist mit einem
Torsionsdraht mit Richtmoment $D$ verbunden und liegt in einem (zumindest die
beiden Hälften) isotropen Magnetfeld.

Die Drehmomente, die auftreten, sind das rückstellende Moment des Drahtes, die
Reibung, das Drehmoment durch den Strom sowie den Drehmoment durch den
Induktionsstrom. Dabei ist $G := abnB$ eine Konstante des Galvanometers. Alle
diese Kräfte beschleunigen den Rotor.
\[
	N
	= \Theta \ddot \phi
	= - D \phi - \rho \dot \phi + G I - \frac{G^2}{R_g + R_a} \dot \phi
\]

Für eine konstante Stromstärke $I$ vereinfacht sich die Differentialgleichung
zu:
\[ \phi = \underbrace{\frac GD}_{c_I} I \]

Diese hängt jetzt nur noch vom Strom und von der Empfindlichkeit $c_I$ ab.

Um die Ruhelage schwingt der Rotor harmonisch, mit den Parametern:
\[
	2 \beta = \frac 1 \Theta \left( \rho + \frac{G^2}{R_a + R_g} \right)
	, \quad
	\omega_0^2 = \frac D \Omega
\]

Damit es genau zu einer kritisch gedämpften Schwingung kommt, muss der
Außenwiederstand entsprechend gewählt werden:
%
\begin{equation}
	\label{eq:243.17}
	R_a = \frac{G^2}{2 \sqrt{\Theta D} - \rho} - R_g =: R_\text{Gr}
\end{equation}

Über einem Widerstand entlädt sich ein Kondensator exponentiell, da der
fließende Strom proportional zur noch verbleibenden Ladung ist.

Wenn man eine sehr kurze Stromspitze misst, so wird die komplette Kraft in
Bewegungsenergie umgewandelt, da die Reibung und rückstellende Kraft erst
später groß wird. Damit lässt sich aus der kinetischen Energie die Gesamtladung
messen.

Am Maximalausschlag $\phi_m$ ist die gesamte kinetische Energie in potentielle
Energie (im Draht) umgewandelt worden. Dann gilt:
\[ \phi_m = \frac G{\sqrt{D \Theta}} Q \]

%%%%%%%%%%%%%%%%%%%%%%%%%%%%%%%%%%%%%%%%%%%%%%%%%%%%%%%%%%%%%%%%%%%%%%%%%%%%%%%
%                                  Aufgaben                                   %
%%%%%%%%%%%%%%%%%%%%%%%%%%%%%%%%%%%%%%%%%%%%%%%%%%%%%%%%%%%%%%%%%%%%%%%%%%%%%%%

\section{Aufgaben}

\subsection{Aufgabe A: Kräfte}

Die Kräfte auf die Spule sind reine Lorentzkräfte. Je nach Lage der Spule sind
diese allerdings in der Spulenebene und erzeugen so kein Drehmoment.
Sinnigerweise ist die Ruhelage so, dass Spulenebene und Magnetfeld parallel
sind. Die Auslenkung aus der Ruhelagen nenne ich $\phi$. Somit ist die Kraft
bei einer Spulenhöhe $a$, einer Spulenbreite $b$ und einem Strom $I$:
\[ N_e = 2 n a I B \half b = abIBn \]

Dabei tragen nur die beiden Stücke, die senkrecht zum Magnetfeld sind, zur
Kraft bei. Die beiden (oder eins und zwei halbe) Stücke, die parallel zum
Magnetfeld sind, tragen kein Drehmoment bei.

Es gibt keinen $\cos(\phi)$ Term, weil das Magnetfeld durch den Weicheisenkern
isotrop gemacht wird.

\subsection{Aufgabe B: induzierte Spannung}

Das Induktionsgesetz besagt:
\[ U = - \dot \Phi = - \od{}{t} n B A = n \left( \dot B A + B \dot A \right) \]

Hier ist \( A(t) = \hat A \cos(\omega t) \) und somit \( \dot A(t) = - \hat A
\omega \sin(\omega t) \). Daraus folgt direkt, dass
\[
	\hat U := U \csc(\omega t) \propto \omega.
\]

\subsection{Aufgabe C: magnetischer Fluss}

Hier gilt:
\[ \Phi = n A B = n a b B \]

Es geht also ein:

\begin{enumerate}
	\item Breite $b$
	\item Höhe $a$
	\item Wicklungszahl $n$
	\item magnetische Flussdichte $B$
\end{enumerate}

\subsection{Aufgabe D: Selbstinduktion}

Die Spule mit der Induktivität $L$ wird bei einer Stromänderung $\dot I$ noch
eine Spannung $U = - L \dot I$ erzeugen, durch die dann wieder ein Strom \[ I_L
= - \frac{L \dot I}{R_g + R_a} \] erzeugt wird. Wir gehen wohl davon aus, dass
der außen anliegende Strom relativ konstant ist und/oder dass die Induktivität
klein ist.

Der Term würde lauten:
\[ - \frac{L G \dot I}{R_g + R_a} \]

\subsection{Aufgabe E: ebene Polschuhe}

Ohne den Weicheisenkern ist das Magnetfeld einfach homogen, und somit kommt
noch ein $\cos(\phi)$ Term hinein.

\subsection{Aufgabe F: Dimensionen}

Die Einheit von $\Theta$ ist $\unit{kg \cdot m^2}$. Die Dämpfungskonstante
$\rho$ hat die Einheit $\unitfrac{N \cdot m \cdot s}{rad}$. Zusammen mit
$\unit{N} = \unitfrac{kg \cdot m}{s^2}$ ergibt sich als Einheit für $\beta$
$\unitfrac{rad}s$.

Die Richtkonstante $D$ hat die Einheit $\unitfrac{N \cdot m}{rad}$. Durch
$\Theta$ bleibt dann $\unitfrac{rad^2}{s^2}$. Die Wurzel daraus ist $\omega_0$
mit der Einheit $\unitfrac{rad}{s}$. Lässt man das $\unit{rad}$ weg, ist es
eine inverse Zeit.

\subsection{Aufgabe G: Grenzwiderstand}

Wenn man $R_\text{Gr}$ kennt, kann man den Widerstand des ganzen Messgerätes so
wählen, dass es möglichst schnell zur Ruhe kommt. Man wählt einen leicht
größeren Widerstand, damit es noch etwas ruhiger schwingt und nicht die
langsamer zerfallende Komponente $t \exp(\lambda t)$ auftritt.

\subsection{Aufgabe H: Empfindlichkeit}

Man kann die Empfindlichkeit steigern, indem man …

\begin{itemize}
	\item … die Anzahl der Windungen erhöht.
	\item … die magnetische Flussdichte erhöht.
	\item … die Fläche der Spule erhöht.
	\item … einen weniger starren Draht verwendet.
\end{itemize}

\subsection{Aufgabe I: Grenzen}

Irgendwann stößt man an Grenzen. Etwa, weil die Spule zu klobig wird oder das
Feld nicht mehr zu erhöhen ist. Oder der Draht ist derart weich, dass schon
kleinste Erschütterungen die Spule auslenken.

\subsection{Aufgabe J: Einheit}

Da \( \phi = c_I I \) gelten soll, sollte die Einheit von $c_I$
$\unitfrac{rad}{A}$ sein.

\subsection{Aufgabe K: andere Messverfahren}

Zum Messen von Widerständen können wir dies einfach mit einem Multimeter
messen. Wenn der Widerstand allerdings zu groß ist, fließt fast kein messbarer
Strom.

Die Wheatstonesche Brücke funktioniert auch nur, wenn man auf beiden Seiten
etwa gleich große Widerstände hat. Dann fließt aber auch wieder fast kein
Strom, was ebenfalls ein extrem empfindliches Messgerät erfordert.

%%%%%%%%%%%%%%%%%%%%%%%%%%%%%%%%%%%%%%%%%%%%%%%%%%%%%%%%%%%%%%%%%%%%%%%%%%%%%%%
%                          Aufbau und Durchführung                           %
%%%%%%%%%%%%%%%%%%%%%%%%%%%%%%%%%%%%%%%%%%%%%%%%%%%%%%%%%%%%%%%%%%%%%%%%%%%%%%%

\section{Aufbau und Durchführung}

\subsection{Aufgabe a: Dämpfung}

\label{Durchführung-a}

Zuerst betrachten wir nur ein Galvanometer. Diese können wir kurzschließen.

Wir drehen die Spule des Galvanometers vorsichtig, einmal mit und einmal ohne
kurzgeschlossenen äußeren Stromkreis.

\subsection{Aufgabe b: Grenzwiderstand}

Wir schließen das Galvanometer aus \ref{Durchführung-a} an einen Widerstand
(Stöpselwiderstand). Dann variieren wir den Widerstand so lange, bis das
Galvanometer bei manueller Auslenkung den aperiodischen Grenzfall erreicht hat.

Den Widerstand messen wir:
\[ R_\text{Gr} = \unit[\emesswert] \Omega \]

Nun bringen wir Zusatzgewichte an und messen erneut.

\begin{table}[h!]
	\centering

	\begin{tabular}{cc}
		Gewicht $m$ in $\unit{kg}$ & Grenzwiderstand $R_\text{Gr}$ in $\unit \Omega$ \\
		\hline
		$\messwert$ & $\emesswert$ \\
		\vdots & \vdots
	\end{tabular}
\end{table}

Dies wird in \ref{Auswertung-b} erklärt.

\subsection{Aufgabe c: linearer Zusammenhang}

\label{Durchführung-c}

Nun bauen wir einen Spannungsteiler mit $R_1$ und $R_2$ an einer Stromquelle
$U_0$ auf. An den Spannungsteiler schließen wir in Reihe einen Regelwiderstand
$R$ und ein Galvanometer an, letzteres können wir mit einem Taster überbrücken.

\textcolor{blue}{Zeichnung einfügen.}

\[
	U_0 = \unit[\emesswert] V
	, \quad
	R_1 = \unit[\emesswert] \Omega
	, \quad
	R_2 = \unit[\emesswert] \Omega
\]

Wir variieren $R$ und messen $\phi$.

\begin{table}[h!]
	\centering

	\begin{tabular}{cc}
		$R$ in $\unit \Omega$ & $\phi$ in $\unit{rad}$ \\
		\hline
		$\messwert$ & $\messwert$ \\
		\vdots & \vdots
	\end{tabular}
\end{table}

Dabei sind die Fehler:
\[
	\Delta R = \unit[\messwert] \Omega
	, \quad
	\Delta \phi = \unit[\messwert]{rad}
\]

Diese Daten werte ich in \ref{Auswertung-d} aus.

\subsection{Aufgabe f: Widerstand direkt gemessen}

Wir messen den Widerstand $R_g$ aus \ref{Durchführung-c} der Galvanometerspule
direkt mit einem dafür geeigneten Messgerät:
\[ R_g = \unit[\emesswert] \Omega \]

\subsection{Aufgabe g: verschiedene Zusatzgewichte}

Wir messen den Ausschlag $\phi$ des Galvanometers im Aufbau aus
\ref{Durchführung-c} für ein bestimmtes $R$ mit verschiedenen Zusatzgewichten.

\begin{table}[h!]
	\centering

	\begin{tabular}{cc}
		$m$ in $\unit{kg}$ & $\phi$ in $\unit{rad}$ \\
		\hline
		$\messwert$ & $\messwert$ \\
		\vdots & \vdots
	\end{tabular}
\end{table}

\subsection{Aufgabe h: großer Widerstand}

Wir vermessen einen großen Widerstand. Dazu laden wir den Kondensator $C$
zuerst auf. Dann entladen wir ihn eine Zeit $t$ über dem unbekannten Widerstand
$R_x$. Anschließend messen wir die Restladung mit dem Galvanometer.

\textcolor{blue}{Skizze einfügen}

\begin{table}[h!]
	\centering

	\begin{tabular}{cc}
		$t$ in $\unit{s}$ & $\phi$ in $\unit{rad}$ \\
		\hline
		$\messwert$ & $\messwert$ \\
		\vdots & \vdots
	\end{tabular}
\end{table}

%%%%%%%%%%%%%%%%%%%%%%%%%%%%%%%%%%%%%%%%%%%%%%%%%%%%%%%%%%%%%%%%%%%%%%%%%%%%%%%
%                                 Auswertung                                  %
%%%%%%%%%%%%%%%%%%%%%%%%%%%%%%%%%%%%%%%%%%%%%%%%%%%%%%%%%%%%%%%%%%%%%%%%%%%%%%%

\section{Auswertung}

\subsection{Aufgabe b: Grenzwiderstand}

\label{Auswertung-b}

Laut Gleichung \eqref{eq:243.17} hängt der Grenzwiderstand von $\Theta$ ab,
welches sich durch das Anbringen von Zusatzgewichten erhöht. Somit wird der
Grenzwiderstand kleiner.

\subsection{Aufgabe d: Stromempfindlichkeit}

\label{Auswertung-d}

Ich fitte die Daten aus \ref{Durchführung-c} mit:
\[
	\frac 1 \phi
	= \underbrace{\frac{R_1 + R_2}{c_I U_0 R_2}}_{\alpha} (R_g + R)
\]

Ich erhalte:
\[
	\alpha = \emesswert
	, \quad
	R_g = \unit[\emesswert] \Omega
\]

Daraus errechne ich:
\[
	c_I = \frac{R_1 + R_2}{\alpha U_0 R_2}
\]

Und den Fehler:
\[
	\Delta c_I
	= \sqrt{
		\left( \frac{R_1 + R_2}{\alpha^2 U_0 R_2} \Delta \alpha \right)^2
		+ \left( \frac{1}{\alpha U_0 R_2} \Delta R_1 \right)^2
		+ \left( \frac{\alpha U_0 R_2 - (R_1 + R_2)\alpha U_0}{\alpha^2 U_0^2 R_2^2} \Delta R_2 \right)^2
		+ \left( \frac{R_1 + R_2}{\alpha U_0^2 R_2} \Delta U_0 \right)^2
	}
\]

Ich erhalte:
\[ c_I = \unitfrac[\emesswert]{rad}{A} \]

\subsection{Aufgabe e: Widerstand der Galvanometerspule}

Aus dem Fit in \ref{Auswertung-d} ist bereits $R_g$ herausgefallen:
\[
	R_g = \unit[\emesswert] \Omega
\]

\subsection{Aufgabe f: Widerstand direkt gemessen}

Die beiden Werte liegen $\messwert$ auseinander.

\subsection{Aufgabe g: verschiedene Zusatzgewichte}

Der Ausschlag hat sich nicht geändert. Dadurch, dass nur die Geschwindigkeit
vom Trägheitsmoment, nicht aber der Ausschlag davon abhängt, spielen die
Gewichte beim Maximalausschlag keine Rolle.

\subsection{Aufgabe h: großer Widerstand}

In der Anleitung stand noch die Frage, warum wir $\phi_m$ nicht in die
Restladung $Q$ umrechnen müssen. Die Antwort steckt in
\[ \phi_m = \frac{G}{\sqrt{D \Theta}} Q. \]

Durch die halblogarithmische Darstellung wird der Vorfaktor zu einer additiven
Konstante, die die Steigung der Geraden nicht beeinflusst. Daher kann diese
Vernachlässigt werden.

Ich fitte die Daten $\phi_m(t)$ mit der Funktion:
\[ \phi_m(t) = Q_0 \exp \left( - \frac{t}{RC} \right) \]

Dabei erhalte ich aus dem Fit:
\[
	RC = \unit[\emesswert] s
	, \quad
	R = \unit[\messwert] \Omega
	, \quad
	\Delta R = \sqrt{
		\left( \frac{1}{C} \Delta R \right)^2
		+ \left( \frac{R}{C^2} \Delta C \right)^2
	} = \unit[\messwert] \Omega
\]

%%%%%%%%%%%%%%%%%%%%%%%%%%%%%%%%%%%%%%%%%%%%%%%%%%%%%%%%%%%%%%%%%%%%%%%%%%%%%%%
%                                  Resultat                                   %
%%%%%%%%%%%%%%%%%%%%%%%%%%%%%%%%%%%%%%%%%%%%%%%%%%%%%%%%%%%%%%%%%%%%%%%%%%%%%%%

\section{Resultat}

Wir haben den Grenzwiderstand des Galvanometers bestimmt auf:
\[ R_\text{Gr} = \unit[\emesswert] \Omega \]

Wir haben außerdem den Innenwiderstand und die Stromempfindlichkeit bestimmt
zu:
\[
	R_g = \unit[\emesswert] \Omega
	, \quad
	c_I = \unitfrac[\emesswert]{rad}{A}
\]

Den Innenwiderstand haben wir außerdem direkt bestimmt zu:
\[ R_g = \unit[\emesswert] \Omega \]

Wir haben festgestellt, dass der Ausschlag bei einem bestimmten Strom nicht von
Zusatzgewichten abhängt.

Einen großen Widerstand haben wir vermessen auf:
\[ R_x = \unit[\emesswert] \Omega \]

%%%%%%%%%%%%%%%%%%%%%%%%%%%%%%%%%%%%%%%%%%%%%%%%%%%%%%%%%%%%%%%%%%%%%%%%%%%%%%%
%                                 Diskussion                                  %
%%%%%%%%%%%%%%%%%%%%%%%%%%%%%%%%%%%%%%%%%%%%%%%%%%%%%%%%%%%%%%%%%%%%%%%%%%%%%%%

\section{Diskussion}

%\bibliography{../../zentrale_BibTeX/Central}
%\bibliographystyle{plain}

%%%%%%%%%%%%%%%%%%%%%%%%%%%%%%%%%%%%%%%%%%%%%%%%%%%%%%%%%%%%%%%%%%%%%%%%%%%%%%%
%                                    Plots                                    %
%%%%%%%%%%%%%%%%%%%%%%%%%%%%%%%%%%%%%%%%%%%%%%%%%%%%%%%%%%%%%%%%%%%%%%%%%%%%%%%

\newpage

\section{Plots}

\begin{figure}[h!]
	\centering
	\includegraphics[height=0.4\textheight]{plot-empfindlichkeit.pdf}
	\caption{Plot der inversen Auslenkung gegen den Widerstand}
\end{figure}

\begin{figure}[h!]
	\centering
	\includegraphics[height=0.4\textheight]{plot-widerstand.pdf}
	\caption{Plot der Auslenkung gegen die Zeit}
\end{figure}

\begin{table}[h!]
	\centering
	\begin{tabular}{rr|rrr}
		$R$ in $\unit{\Omega}$ & $\phi$ & $R$ & $1/\phi$ & $\Delta (1/\phi) = 1/(\phi^2) \Delta \phi$ \\
		\hline
		\input{c.tex}
	\end{tabular}
\end{table}

\begin{table}[h!]
	\centering
	\begin{tabular}{rr|rrr}
		$t$ in $\unit{s}$ & $\phi$ & $t$ & $\ln(\phi)$ & $\Delta(\ln(\phi)) = |\ln(\phi)| \Delta \phi$ \\
		\hline
		\input{h.tex}
	\end{tabular}
\end{table}

\end{document}

% vim: spell spelllang=de
