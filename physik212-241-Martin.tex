% Copyright © 2012 Martin Ueding <dev@martin-ueding.de>
%
\documentclass[11pt, ngerman]{article}

\usepackage[a4paper, left=3cm, right=2cm, top=2cm, bottom=2cm]{geometry}
\usepackage[activate]{pdfcprot}
\usepackage[cdot, squaren]{SIunits}
\usepackage[iso]{isodate}
\usepackage[parfill]{parskip}
\usepackage[T1]{fontenc}
\usepackage[utf8]{inputenc}
\usepackage{amsmath}
\usepackage{amssymb}
\usepackage{amsthm}
\usepackage{babel}
\usepackage{color}
\usepackage{commath}
\usepackage{epstopdf}
\usepackage{graphicx}
\usepackage{hyperref}
\usepackage{setspace}
\usepackage{tikz}

\usepackage[charter]{mathdesign}

\definecolor{darkblue}{rgb}{0,0,.5}
\definecolor{darkgreen}{rgb}{0,.5,0}

\hypersetup{
	breaklinks=false,
	citecolor=darkgreen,
	colorlinks=true,
	linkcolor=black,
	menucolor=black,
	urlcolor=darkblue,
}

\setlength{\columnsep}{2cm}

\DeclareMathOperator{\arcsinh}{arsinh}
\DeclareMathOperator{\arsinh}{arsinh}
\DeclareMathOperator{\asinh}{arsinh}
\DeclareMathOperator{\card}{card}
\DeclareMathOperator{\diam}{diam}

\newcommand{\emesswert}{\left(\messwert \pm \messwert \right)}
\newcommand{\e}[1]{\cdot 10^{#1}}
\newcommand{\fehlt}{\textcolor{red}{Hier fehlen noch Inhalte.}}
\newcommand{\half}{\frac{1}{2}}
\newcommand{\laplace}{\vnabla^2}
\newcommand{\messwert}{\textcolor{blue}{\square}}
\newcommand{\vnabla}{\vec{\nabla}}

\title{physik212 -- Versuch 241 \\ Wechselstromwiderstände, Phasenschieber, RC-Glieder und Schwingungen}
\author{Martin Ueding \\ {\small \href{mailto:mu@uni-bonn.de}{mu@uni-bonn.de}}}

\begin{document}

\maketitle

\tableofcontents

\vfill

\begin{small}
	Stellen in \textcolor{blue}{blau} werden bei der Versuchsdurchführung
	eingetragen. Zum Beispiel werden Messwerte mit „$\messwert$“ markiert.
	Stellen in \textcolor{red}{rot} müssen noch vor Versuchsbeginn
	vervollständigt werden.
\end{small}

\newpage

%%%%%%%%%%%%%%%%%%%%%%%%%%%%%%%%%%%%%%%%%%%%%%%%%%%%%%%%%%%%%%%%%%%%%%%%%%%%%%%
%                                 Einleitung                                  %
%%%%%%%%%%%%%%%%%%%%%%%%%%%%%%%%%%%%%%%%%%%%%%%%%%%%%%%%%%%%%%%%%%%%%%%%%%%%%%%

\section{Einleitung}

In diesem Versuch vermessen wir Wechselstromwiderstände, erzeugen
Lissajousfiguren mit einem Phasenschieber untersuchen frequenzabhängige
Spannungsteiler sowie den elektrischen Schwingkreis.

%%%%%%%%%%%%%%%%%%%%%%%%%%%%%%%%%%%%%%%%%%%%%%%%%%%%%%%%%%%%%%%%%%%%%%%%%%%%%%%
%                                   Theorie                                   %
%%%%%%%%%%%%%%%%%%%%%%%%%%%%%%%%%%%%%%%%%%%%%%%%%%%%%%%%%%%%%%%%%%%%%%%%%%%%%%%

\section{Theorie}

\subsection{Messen von Wechselstromwiderständen}

Kapazitäten können mit einer Art Wheatstoneschen Brücke vermessen werden, dabei
wird ein Oszillograph als Nullinstrument verwendet. Die Abgleichbedingung
lautet für den verlustfreien Kondensator:
\[ \frac{R_1}{R_2} = \frac{Z_0}{Z_x} = \frac{iC_x}{iC_0} \]

Für Induktivitäten geht dies ähnlich, allerdings benötigt man ein zweites
Potentiometer, damit man die Widerstände der Spulen ausgleichen kann. Die
Abgleichbedingung lautet:
\[ \frac{R_1}{R_2} = \frac{L_x}{L_0} = \frac{R_x + R'}{R_0 + R''} \]

\subsection{Phasenschieber}

Ein Phasenschieber kann eine Spannung von $\approx \unit0\rad$ bis $\approx
\unit\pi\rad$ verschieben, ohne dass die Amplitude sich verändert. Dazu muss
allerdings $R_1 = R_2$ gelten.

\textcolor{red}{Hier Zeichnung abmalen.}

\subsection{Schwingkreis}

Der (angetriebene) Schwingkreis entspricht dem (angetriebenen) Drehpendel der
Mechanik (siehe \ref{Aufgabe D}). Die Spannungen an den jeweiligen Bauteilen
können als Funktionen der Ladung (oder Strom $I=\dot q$) ausgedrückt werden:
\[ L \ddot q + R \dot q + \frac 1c q = U_E \cos(\omega t) \]

Analog zum Drehpendel gibt es die Schwingungsgrößen:
\[
	\omega_0^2 = \frac{1}{LC}
	, \quad
	Q = \omega_0 \frac LC
	, \quad
	\omega_\text{max} = \omega_0 \sqrt{1 - \frac{1}{2 Q^2}}
\]

%%%%%%%%%%%%%%%%%%%%%%%%%%%%%%%%%%%%%%%%%%%%%%%%%%%%%%%%%%%%%%%%%%%%%%%%%%%%%%%
%                                  Aufgaben                                   %
%%%%%%%%%%%%%%%%%%%%%%%%%%%%%%%%%%%%%%%%%%%%%%%%%%%%%%%%%%%%%%%%%%%%%%%%%%%%%%%

\section{Aufgaben}

\subsection{Aufgabe A}

Mit den zusätzlichen Widerständen $R'$ und $R''$ werden die Widerstände $R_x$
und $R_0$ so verlängert, dass $L_x$ und $L_0$ das gleiche Verhältnis wie die
Widerstände haben.

\textcolor{red}{Phasendiagramm einfügen.}

\subsection{Aufgabe B}

Die beiden parallelen Pfade werden über die Spannung gleichgesetzt. Somit
müssen die Gesamtspannung beider Teile entlang $U_E$ liegen. Zwischen den
Punkten $A$ und $B$ liegt die Ausgangsspannung $U_{AB}$.

Wenn $R_1 \neq R_2$, dann funktioniert das ganze weiterhin, allerdings ist
$\abs{U_{AB}}$ dann abhängig vom Winkel $\phi$.

Das ganze wird mit zwei Spulen oder Kondensatoren nicht so direkt
funktionieren.

Für $\phi \approx \unit 0 \rad$ muss $R \gg \frac 1{\omega C}$ und für $\phi
\approx \unit \pi \rad$ muss $R \ll \frac 1{\omega C}$ gelten. Somit sollte $R$
zwischen $0.1 \frac 1{\omega C}$ und $10 \frac 1{\omega C}$ verstellbar sein.

Ein einzelner Kondensator oder Spule in Reihe können die Phase auch
verschieben. Dabei ist die Amplitude jedoch von $C$ oder $L$ abhängig und man
kann nur bis $\phi \approx \frac \pi 2$ schieben.

\subsection{Aufgabe C}

Der Maximalstrom im $R$-$C$-Zweig ist:
\[
	I_2 = \frac{U_E}{R} \tan(\theta)
	, \quad
	\tan(\theta) = \frac{|Z_C|}{|Z_R|} = \frac 1{\omega C R}
\]

\subsection{Aufgabe D}

\label{Aufgabe D}

Die Differentialgleichung für das Drehpendel:
\[ I \ddot \phi + r \dot \phi + D \phi = F_0 \cos(\omega t) \]

\subsection{Aufgabe E}

\[
	L \triangleq I
	, \quad
	R \triangleq r
	, \quad
	\frac 1C \triangleq D
	, \quad
	U_E \triangleq F_0
\]

Die Auslenkung $\phi$ entspricht der Ladung $q$.

%%%%%%%%%%%%%%%%%%%%%%%%%%%%%%%%%%%%%%%%%%%%%%%%%%%%%%%%%%%%%%%%%%%%%%%%%%%%%%%
%                    Aufbau, Durchführung und Auswertung                     %
%%%%%%%%%%%%%%%%%%%%%%%%%%%%%%%%%%%%%%%%%%%%%%%%%%%%%%%%%%%%%%%%%%%%%%%%%%%%%%%

\section{Aufbau, Durchführung und Auswertung}

Ohne Trenntrafo würde es zu einem Kurzschluss kommen, wenn $R_1 = \unit0\ohm$
eingestellt wird. Dann würde Strom von einem Pol des Sinusgenerators zum
Potentiometerabgriff, von dort zum Oszillograph, dort in die Erde und von der
Erde wieder zum Sinusgenerator fließen, ohne einen Widerstand dazwischen.

\subsection{Messung von Wechselstromwiderständen}

\subsubsection{Aufgabe a}

\label{Aufgabe a}

\paragraph{Aufbau}

Wir benutzen eine Art Wheatstonesche Brücke, um die Kapazität eines
Kondensators zu bestimmen. Dazu benutzen wir einen bekannten Kondensator mit
$C_0 = \unit \emesswert \farad$.

\paragraph{Durchführung}

Wir verstellen das Helipot (1000 Skalenteile), bis auf dem Oszillographen ein
verschwindender Ausschlag angezeigt wird. Dies ist bei $x = \messwert$
Skalenteilen erreicht.

\paragraph{Auswertung}

Die Abgleichbedingung liefert:
\begin{equation}
	\label{eq:a}
	C_x = \frac{x}{1000} C_0 = \unit \emesswert \farad
\end{equation}

\subsubsection{Aufgabe b}

\label{Aufgabe b}

\paragraph{Aufbau}

Wie in \ref{Aufgabe a} verwenden wir die Wheatstonesche Brücke, diesmal
allerdings mit zwei Spulen und einem zweiten Helipot ($\messwert$ Skalenteile).
Dabei hat die bekannte Spule eine Induktivität von $L_0 = \unit \emesswert
\henry$.

\paragraph{Durchführung}

Wir stellen wieder das erste Helipot so ein, dass keine Spannung mehr fließt.
Dies ist bei $x = \messwert$ erreicht.

\paragraph{Auswertung}

Analog zu \ref{Aufgabe a} liefert die Abgleichbedingung:

\begin{equation}
	\label{eq:b}
	L_x = \frac{x}{1000} L_0 = \unit \emesswert \henry
\end{equation}

\subsubsection{Aufgabe c}

\label{Aufgabe c}

\paragraph{Aufbau}

Wir schalten die Spule aus \ref{Aufgabe b} mit einem Spannungsmesser parallel
und das ganze hinter einen Strommesser.

\paragraph{Durchführung}

Wir messen den Spulenwiderstand mit einem Digitalmultimeter auf: $R = \unit
\emesswert \ohm$. Dann messen wir für eine gegebene Eingangswechselspannung
($\omega = \unit \emesswert \radianpersecond)$) Spannung und Strom:
\[
	U = \unit \emesswert \volt
	, \quad
	I = \unit \emesswert \ampere
\]

Den Innenwiderstand der Messgeräte bestimmen wir auf:
\[
	R_U = \unit \emesswert \ohm
	, \quad
	R_I = \unit \emesswert \ohm
\]

\paragraph{Auswertung}

Das Zeigerdiagramm ist in Abbildung \ref{fig:Aufgabe c} dargestellt. Dabei ist
der Strom $I$ die reelle Achse. \textcolor{red}{Einwirkung der Messgeräte.}

\begin{figure}[h!]
	\centering
	\begin{tikzpicture}
		\draw[->] (0, 0) -- (6, 0) node[label=90:$U_R$] {};
		\draw[->] (0, 0) -- (0, 4) node[label=0:$U_L$] {};
		\draw[->] (0, 0) -- (6, 4) node[label=0:$U$] {};
	\end{tikzpicture}
	\caption{Zeigerdiagramm für \ref{Aufgabe c}}
	\label{fig:Aufgabe c}
\end{figure}

Im Idealfall stehen Strom und Spannung folgendermaßen in Beziehung:
\[ U = \sqrt{R^2 + \del{\omega L}^2} I \]

Dies kann ich auflösen:
\[
L = \sqrt{\frac 1\omega \del{\del{\frac UI}^2 - R^2}}
= \unit \emesswert \henry
\]

Aus dem Zeigerdiagramm folgt für die Phasenverschiebung:
\[
	\tan\del\phi = \frac{\abs{\omega L}}{\abs{U_R}}
	, \quad
	\tan = \unit \emesswert \rad
\]

\subsection{Phasenschieber}

\subsubsection{Aufgabe d}

\paragraph{Aufbau}

Wir benutzen den Phasenschieber aus der Aufgabenstellung.

\textcolor{blue}{Hier die Zeichnung aus der Aufgabenstellung abmalen.}

Dabei lassen wir auf der einen Achse des Oszillographen die Eingangsspannung,
auf der anderen Achse die Ausgangsspannung anzeigen.

\paragraph{Durchführung}

Wir variieren $R$ von $0$ bis $R_\text{max}$ und messen $U_R$ und $U_C$.
Zuletzt entfernen wir den Widerstand $R$ um $R = \infty$ zu erreichen.

\begin{table}[h!]
	\centering
	\begin{tabular}{ccc}
		$R$ & $U_R$ & $U_C$ \\
		\hline
		$\messwert$ & $\messwert$ & $\messwert$ \\
			 \vdots & 	 \vdots & \vdots \\
		$\infty$ & $\messwert$ & $\messwert$ \\
	\end{tabular}
\end{table}

\paragraph{Auswertung}

Ich zeichne ein maßstabsgetreues Zeigerdiagramm mit $U_E$ (Abbildung
\ref{fig:Aufgabe d}). Jedes Paar $U_R$ und $U_C$ zeichne ich ein.

Auf dem Oszillographen sehen wir diverse Lissajousfiguren. Dabei ist die
Frequenz bei beiden Eingängen gleich, die Phase allerdings verschieden. Somit
sind alle Figuren nur eine Ellipse.

\paragraph{Lissajous-Figur}

Eine Lissajous-Kurve $\mathcal L$ ist:
\[
	\mathcal L\del{\hat x, \hat y, \omega_x, \omega_y, \phi_x, \phi_y} =
	\set{
		(x, y) \in \mathbb R^2
		\colon x = \hat x \cos\del{\omega_x t + \phi_x}
		\wedge y = \hat y \cos\del{\omega_y t + \phi_y}
		\wedge t \in \mathbb R
	}
\]

Wenn $\frac{\omega_x}{\omega_y}$ eine rationale Zahl ist, ist die Kurve
geschlossen und füllt nicht den gesamten Raum aus.

Auf einem Oszillographen kann man diese sichtbar machen, in dem man einen
Sinusgenerator an $x$, einen zweiten an $y$ anschließt.

\subsection{frequenzabhängige Spannungsteiler}

\subsubsection{Aufgabe e}

\paragraph{Aufbau}

Wir bauen nacheinander einen Tiefpass, einen Hochpass und einen Sperrfilter
auf. An den Eingang schließen wir einen Signalgenerator, an den Ausgang hängen
wir ein Spannungsmessgerät.

Der Widerstand ist $R = \unit{100}\ohm$, die Kapazität ist $C = \unit{1.5}{\micro\farad}$.

\paragraph{Durchführung}

Für jede Schaltung messen wir die Ausgangsspannung in Abhängigkeit von der
Eingangsfrequenz $\nu \in \unit{[200, 5000]}\hertz$. Dabei machen wir für jede
Schaltung 11 Messpunkte, so dass die Eingangsspannung exponentiell wächst.
Dabei halten wir die Eingangsspannung konstant.

\begin{table}[h!]
	\centering
	\begin{tabular}{cccc}
		$\nu$ & $U_A$ & $U_A$ & $U_A$ \\
		\hline
		$\unit{ 200}\hertz$ & $\messwert$ & $\messwert$ & $\messwert$ \\
		$\unit{ 275}\hertz$ & $\messwert$ & $\messwert$ & $\messwert$ \\
		$\unit{ 380}\hertz$ & $\messwert$ & $\messwert$ & $\messwert$ \\
		$\unit{ 525}\hertz$ & $\messwert$ & $\messwert$ & $\messwert$ \\
		$\unit{ 724}\hertz$ & $\messwert$ & $\messwert$ & $\messwert$ \\
		$\unit{1000}\hertz$ & $\messwert$ & $\messwert$ & $\messwert$ \\
		$\unit{1379}\hertz$ & $\messwert$ & $\messwert$ & $\messwert$ \\
		$\unit{1903}\hertz$ & $\messwert$ & $\messwert$ & $\messwert$ \\
		$\unit{2626}\hertz$ & $\messwert$ & $\messwert$ & $\messwert$ \\
		$\unit{3623}\hertz$ & $\messwert$ & $\messwert$ & $\messwert$ \\
		$\unit{5000}\hertz$ & $\messwert$ & $\messwert$ & $\messwert$ \\
	\end{tabular}
\end{table}

\paragraph{Auswertung}

Für die Diagramme normiere ich die Frequenzen und Ausgangsspannungen wir folgt.
Für Tief- (Abbildung \ref{fig:Aufgabe e Tiefpass}) und Hochpass (Abbildung
\ref{fig:Aufgabe e Hochpass}) wird die Abszisse auf $\Omega = \nu R C$ normiert
und die Ordinate auf $A = \frac{U_A}{U_E}$.

Für den Sperrfilter (Abbildung \ref{fig:Aufgabe e Sperrfilter}) wir die
Abszisse auf $\Omega = \frac{\nu}{\nu_0}$ gesetzt, dabei ist $\nu_0$ beim
Minimum, das bei $\nu_0 = \unit{\messwert}{\hertz}$ liegt.

Für die Ordinate lege ich außerdem noch eine $\deci\bel$-Skala fest. Bei $U_A =
U_E$ sind es $\unit0{\deci\bel}$, bei der Hälfte der Spannungsamplitude
(Viertel der Leistung) sind es nur noch ungefähr $\unit{-6}{\deci\bel}$. Für
den Pegel gilt:
\[ L = \unit{20 \log_{10} \del{\frac{U_A}{U_E}}}{\deci\bel} \]

\subsubsection{Aufgabe f}

Zu Tief- und Hochpass bestimme ich die Grenzfrequenz, die ich aus dem Diagramm
ablese.
\[
	\nu_{\text{gr},\text{TP}} = \unit\emesswert\hertz
	, \quad
	\nu_{\text{gr},\text{HP}} = \unit\emesswert\hertz
\]

Die theoretische Grenzfrequenz sollte sein:
\[
	2 \pi \nu_\text{gr} = \omega_\text{gr} = \frac1{RC}
	, \quad
	\nu_\text{gr} = \unit{1061}\hertz
\]

\subsubsection{Aufgabe g}

Aus dem Diagramm für den Sperrfilter kann ich die Unterdrückungsgüte ablesen.
Dabei ist $\Delta \nu$ die Breite, innerhalb derer $U_A < \frac{U_A}{\sqrt{2}}$
gilt.
\[
	Q'_\text{exp} = \frac{\nu_0}{\Delta \nu} = \frac{\emesswert}{\emesswert} = \emesswert
\]

Der theoretische Wert ist:
\[
	Q'_\text{theo} = \frac{1}{\omega_0 R C} = \messwert
\]

\textcolor{blue}{Die beiden Werte liegen (oder auch nicht) nah beieinander}.

\subsubsection{Aufgabe h}

Wenn die Frequenz, die abgeschwächt werden soll, nicht verändert werden soll,
kann nur noch über den Widerstand die Unterdrückungsgüte eingestellt werden.
Der Spulenwiderstand verschlechtert die Güte zusätzlich.

\subsection{elektrischer Schwingkreis}

\subsubsection{Aufgabe i}

\paragraph{Aufbau}

Wir benutzen einen Seriellschwingkreis aus realer Spule mit Widerstand $R_L$
und einem Kondensator ($C = \unit{1.5}{\micro\farad}$). Der Schwingkreis wird
mit einem Signalgenerator, der über einen Transformator angeschlossen ist,
angetrieben. Über dem Kondensator wird die Spannung $U_C$ abgegriffen. Die
Eingangsspannung $U_E$ messen wir mit einem Spannungsmessgerät.

\paragraph{Durchführung}

Wir messen die Resonanzkurve im Bereich von $\unit0\hertz$ bis
$\unit{2000}\hertz$. Dabei achten wir daraus, dass die Amplitude der
Eingangsspannung $U_E$ konstant bleibt. Außerdem vermessen wir den Resonanzbereich genauer als die Flanken.

\begin{table}[h!]
	\centering
	\begin{tabular}{cc}
		$\nu$ & $U_C$ \\
		\hline
		$\messwert$ & $\messwert$ \\
				\vdots & \vdots
	\end{tabular}
\end{table}

\paragraph{Auswertung}

Ich trage die Messwerte in ein $U_C$ gegen $\nu$ Diagramm (Abbildung \ref{fig:Aufgabe i}).

Aus dem Diagramm bestimme ich $Q$ aus der Resonanzüberhöhung:
\[
	Q = \frac{U_A\del{\omega_\text{max}}}{U_A(0)} = \emesswert
\]

Sowie $Q$ aus der Resonanzbreite:
\[
	Q = \frac{\omega_0}{\Delta \omega} = \emesswert
\]

Außerdem bestimme ich $\omega_\text{max}$ aus $\omega_0$:
\[
	\omega_\text{max} = \omega_0 \sqrt{1 - \frac{1}{2 Q^2}} = \unit\emesswert\radianpersecond
\]

Sowie $L$:
\[
	L = \frac{1}{\omega_0^2 C} = \unit\emesswert\henry
\]

Und die Güte noch einmal:
\[
	Q = \omega_0 \frac{L}{R_L} = \emesswert
\]

%%%%%%%%%%%%%%%%%%%%%%%%%%%%%%%%%%%%%%%%%%%%%%%%%%%%%%%%%%%%%%%%%%%%%%%%%%%%%%%
%                                  Resultat                                   %
%%%%%%%%%%%%%%%%%%%%%%%%%%%%%%%%%%%%%%%%%%%%%%%%%%%%%%%%%%%%%%%%%%%%%%%%%%%%%%%

\section{Resultat}

\fehlt

%%%%%%%%%%%%%%%%%%%%%%%%%%%%%%%%%%%%%%%%%%%%%%%%%%%%%%%%%%%%%%%%%%%%%%%%%%%%%%%
%                                 Diskussion                                  %
%%%%%%%%%%%%%%%%%%%%%%%%%%%%%%%%%%%%%%%%%%%%%%%%%%%%%%%%%%%%%%%%%%%%%%%%%%%%%%%

\section{Diskussion}

\fehlt

%\bibliography{../../zentrale_BibTeX/Central}
%\bibliographystyle{plain}

\newpage

%%%%%%%%%%%%%%%%%%%%%%%%%%%%%%%%%%%%%%%%%%%%%%%%%%%%%%%%%%%%%%%%%%%%%%%%%%%%%%%
%                                    Plots                                    %
%%%%%%%%%%%%%%%%%%%%%%%%%%%%%%%%%%%%%%%%%%%%%%%%%%%%%%%%%%%%%%%%%%%%%%%%%%%%%%%


%\begin{figure}[h]
%	\centering
%	\input{d_zeiger.tex}
%	\caption{Zeigerdiagramm, der erste Pfeil ist immer $U_R$, der zweite Pfeil $U_C$.}
%	\label{fig:Aufgabe d}
%\end{figure}

\begin{figure}[h!]
	\centering
	\includegraphics[height=0.4\textheight]{e_tief.pdf}
	\caption{Tiefpass}
	\label{fig:Aufgabe e Tiefpass}
\end{figure}

\begin{figure}[h!]
	\centering
	\includegraphics[height=0.4\textheight]{e_hoch.pdf}
	\caption{Hochpass}
	\label{fig:Aufgabe e Hochpass}
\end{figure}

%\begin{figure}[h]
%	\centering
%	\includegraphics[height=0.45\textheight]{e_sperr.pdf}
%	\caption{Sperrfilter}
%	\label{fig:Aufgabe e Sperrfilter}
%\end{figure}

%\begin{figure}[h]
%	\centering
%	\includegraphics[height=0.45\textheight]{i.pdf}
%	\caption{Resonanzkurve Schwingkreis}
%	\label{fig:Aufgabe i}
%\end{figure}

\end{document}

% vim: spell spelllang=de
